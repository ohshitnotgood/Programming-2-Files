\section{Cats}
How do our \st{philosophers} cats maintain a work-life balance? Through message passing of course!

Each cat has a food value (how much they need to eat before they are full), a hunger value (when hunger becomes 0, they die), and a laziness value (how long they \st{sleep} think before they must eat again).

A cat makes a request to the kernel and awaits a response (via the thread-blocking \texttt{receive} method). If the request is \texttt{:denied}, hunger is decremented and \texttt{meow/8} is called again to form a recursive loop.
If a request is \texttt{:approved} however, the cat spends some time eating (for 1 millisecond for our demonstration), then spends some time sleeping before requesting the \texttt{Katnel} again for chopsticks.

Notable that the \texttt{receive} block is thread-blocking so our cat does nothing while it waits for an answer from the server. Maybe it should've spend some time doing some thinking while it waited. 

Finally when the food value reaches 0, the cat is done eating and can notify the kernel about it's termination.

\pagebreak
\begin{lstlisting}[language=Elixir, title=How our cat brain works]
def meow(cat, krnlPID, c1, c2, food, base_hngr, hngr, laziness) do
  # request chopsticks
  send(krnlPID, {:request, cat, hngr, c1, c2, self()})

  # thread-blocking
  receive do
    :denied ->
      meow(cat, krnlPID, c1, c2, food, base_hngr, hngr - 1, laziness)

    {:approved, c1, c2} ->
      :timer.sleep(1)

      # return chopsticks
      send(krnlPID, {:return, c1, c2})

      # start thinking 
      :timer.sleep(laziness)

      # thinking over, back to eating
      meow(cat, krnlPID, c1, c2, food - 1,  base_hngr, base_hngr, laziness)
  end
end
\end{lstlisting}

Notable is that with this way of implementation, our algorithm is scalable. Meaning we can add more cats without changing the structure of our algorithm.