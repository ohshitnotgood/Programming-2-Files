\section{Dealing with edge cases}

\subsection{End of list}
There are a handful of edge cases that we must encounter.

First, when happens when we reach the end of the tree?

Well, that's a simple solution for now. The end of the tree will likely have a \texttt{null} node. At that point, we can simply pull from our stack and traverse the node that we have in our stack.

We can also handle the case where are stack is also empty i.e.\ we must end our program.

\subsection{When we have matched a clue}
This is not the case where we have matched a pattern.
Matching a clue is when our clue says we must have three broken springs in a row and we do get three broken springs in a row.

We have been maintaining a \texttt{row} variable for that which we can use to check for that. If the number of rows have been reached, we can reset row to 0, and increment our \texttt{clue\_index}.

\subsection{Clue overflow?}
When happens when the \texttt{clue\_index == length(clue)}? Well, we have already matched our pattern. From now on, we can only accept working hot springs. 

If we encounter any unknown hot spring, we must purposefully take the left branch without bothering with the stack. 

If we encounter a broken hot spring, then our solution so far has not been a correct one and thus we must discard this solution and start over from our latest stack position.

\subsection{Time4Bureaucracy?}
We can assume that we have encountered a correct solution when all of the following are met:
\begin{itemize}
    \item We have encountered a \texttt{nil} node
    \item \texttt{clue\_index == length(clue)}
    \item \texttt{length(result) == length(raw\_pattern)} where \texttt{raw\_pattern} is the string pattern before we parsed it.
\end{itemize}