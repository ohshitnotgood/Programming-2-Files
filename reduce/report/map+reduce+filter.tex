\section{Implementing the map function}

Now that we have discovered the secrets of the French, we can deploy those secrets to our \texttt{map/2} function.

The \texttt{map/2} function uses the secret employed by the, previously described, \texttt{inc/1} function.

\begin{lstlisting}[language=Elixir, caption=Mapping\, recursively]
defmodule Napoleon do
  def map([head | []], lambda) do
    [lambda.(head)]
  end

  def map([head | tail], lambda) do
    [c | t] = tail
    [lambda.(head) | map([c | t], lambda)]
  end
end
\end{lstlisting}

As mentioned earlier, this is nothing but reusing the logic from our \texttt{inc/1} function. Only difference is that our \texttt{inc/1} function only took the list of integers as the only parameter and simply added 1 in every recursion step. 

In our \texttt{map/2} function, we no longer hardcode our arithmetic but instead use a \texttt{lambda} function, which is passed as a parameter, to do the numeric operations for us.


\section{Filtration}
As with the case with \texttt{map/2}, the \texttt{filter/2} function is a reusing of the logic from \texttt{even/1}. Instead of hardcoding our true/false check with the \texttt{rem/2} function, we pass a \texttt{lambda} function to do the function for us, the \texttt{lambda} function returning true/false.

\begin{lstlisting}[language=Elixir, caption=Mapping\, recursively]
defmodule Napoleon do
    def filter([head | []], lambda) do
      case lambda.(head) do
        true -> [head]
        false -> []
      end
    end
  
    def filter([head | tail], lambda) do
      case lambda.(head) do
        true -> [head | filter(tail, lambda)]
        false -> filter(tail, lambda)
      end
    end
end
\end{lstlisting}

We again exploit the feature (or bug whatever it may be. Elixir is a weird language no doubt) that \texttt{{[ 1, 3 | []]}} returns \texttt{{[1, 3]}}


\section{Reduction}
And finally, we implement the \texttt{reduce/3} method: a more versatile version of our \texttt{length/1} method.

Detailed explanation is unnecessary: the hardcoded add operation has merely been replaced by a lambda function that must take in two parameters.


\begin{lstlisting}[language=Elixir, caption=Mapping\, recursively]
defmodule Napoleon do
    def reduce(src, iac, lambda) do
      do_reduce({iac, src}, lambda)
    end

    def do_reduce({x, [head | []]}, lambda) do
      lambda.(x, head)
    end

    def do_reduce({x, [head | tail]}, lambda) do
      do_reduce({lambda.(x, head), tail}, lambda)
    end
end
\end{lstlisting}

\section{But who do we exile?}
The functions, \texttt{even}, \texttt{length}, and \texttt{filter} are tail-recursive meaning the last recursive call merely returns the result without doing any additional processing.

Functions \texttt{map}, \texttt{inc}, and \texttt{reduce}, on the other hand are not, as I would argue, tail recursive as they do some additional processing before returning their results and thus shall be exiled.