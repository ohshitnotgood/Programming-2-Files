\section{Finding the length}

For finding the length recursively, we implement two functions (apart from the function included in the skeleton).

We first use pattern matching to convert our list into a cons cell, the head of with can be discarded as the French used to say during the time of their revolution.

\texttt{r\_length(\{x, []\})} is called when we have traversed to the end of the list and it is time to return our sum. Otherwise we recursively call the other \texttt{r\_length} method.

\begin{lstlisting}[language=Elixir, caption=Length\, recursively]
defmodule Reduce do
    def length(arg) do
      r_length({0, arg})
    end
  
    def r_length({x, []}) do
      x
    end
  
    def r_length({x, [_ | tail]}) do
      r_length({x + 1, tail})
    end
end
\end{lstlisting}

\section{Incrementing each value}
While implementing \texttt{inc/1} we cannot let our elements meet the same fate as Marie-Antoinette: we must keep all the elements without discarding any heads.

We can achieve this by exploiting a feature in Elixir cons cell: if the tail of a cons cell is another cons cell, then the result is a cons cell where the head of the root cell and the head of the tail of the root are appended in to a list and the tail of the root is the new tail. What on earth does that mean?
\pagebreak
\begin{lstlisting}[language=Elixir, caption=Cons cells\, exploited]
iex(1)> [1 | [2 | 3]]
[1, 2 | 3]
\end{lstlisting}

We can thus implement \texttt{inc/1} in the following way:
\begin{lstlisting}[language=Elixir, caption=\texttt{inc/1}\, implemented `exploitedly']
defmodule Reduce do
    def inc([head | []]) do
        [head + 1]
    end

    def inc([head | tail]) do
        [c | t] = tail
        [head + 1 | inc([c | t])]
    end
end
\end{lstlisting}


\section{Filtering out even numbers in our list}
We can implement our \texttt{even/1} method by using logic that is similar to both \texttt{inc/1} and \texttt{length/1}. We must discard the heads that are not even but keep the ones that are. 

Determining if a number is even can be done with the \texttt{rem/2} function. Determining whether to keep our can be done by re-establishing the Jacobin club in the form of a \texttt{case} statement. And thus as it holds

\begin{lstlisting}[language=Elixir, caption=Even numbers\, even-tually]
defmodule Reduce do
    def even([head | []]) do
      case rem(head, 2) == 0 do
        true -> [head]
        false -> []
      end
    end

    def even([head | tail]) do
      case rem(head, 2) == 0 do
        true -> [head | even(tail)]
        false -> even(tail)
      end
    end
end
\end{lstlisting}

Using these tricks, implementing \texttt{div}, \texttt{mul}, and \texttt{odd}  becomes trivial. Hence, code snippets are not included in this report.