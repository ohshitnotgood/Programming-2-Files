\section{Encoding and decoding our text}
Now comes the part to encode and decode our plaintext.

\subsection{Encoding}
Encoding involves taking each character and replacing it with the code from our coding table. That's it.


\subsection{Decoding}
Decoding requires traversal of our tree. We take the first bit of our encoded text and traverse appropriately (left for 0, right for 1). We continue this until we reach a \texttt{:node} at which point we take the next bit but this time we restart from the top of the tree. We pass a \texttt{root} argument which is an unmodified copy of our tree that allows us to begin from the start.

Notice how we pattern match for 48 and 49 as they are ASCII for "0" and "1".

\begin{lstlisting}[language=Elixir, title=Decoding]
defmodule Huffman do  
    def decode [], _, _, decoded_text do
    decoded_text
  end

  def decode [48 | tail], {:leaf, v, l, _}, root, decoded_text do
    case l do
      {:node, k} -> decode(tail, root, root, decoded_text <> k)
      leaf -> decode(tail, leaf, root, decoded_text)
    end
  end

  def decode [49 | tail], {:leaf, v, _, r}, root, decoded_text do
    case r do
      {:node, k} -> decode(tail, root, root, decoded_text <> k)
      leaf -> decode(tail, leaf, root, decoded_text)
    end
  end
  ...
end
\end{lstlisting}