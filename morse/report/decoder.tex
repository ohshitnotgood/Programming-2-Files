\section{The Elixir way}
\begin{figure}[h]
    \centering
    \includegraphics[width=0.5\textwidth]{img/base}
\end{figure}
Our morse code can have three characters: `\texttt{?.}', `\texttt{?-}' and `\texttt{? }'. The period representing a short beep, take right, the dash representing a long beep, take left, the whitespace representing end of character, repeat.

Therefore, we can expand our \texttt{decode/5} method to account for a period
\begin{lstlisting}[language=Elixir]
def decode([?- | rest], {:node, char, left, _}, root, _, msg) do
  decode(rest, left, root, char, msg)
end
\end{lstlisting}
and a dash
\begin{lstlisting}[language=Elixir]
def decode([?. | rest], {:node, char, _, right}, root, _, msg) do
  decode(rest, right, root, char, msg)
end
\end{lstlisting}
and a whitespace
\begin{lstlisting}[language=Elixir]
def decode([32 | rest], {:node, char, _, _}, root, _, msg) do
  msg = msg <> <<char::utf8>>
  decode(rest, root, root, "", msg)
end
\end{lstlisting}

When we encounter a whitespace, of course we have to append our character to the message. We then keep passing this message around as it travels through the recursive path of hell.

However, we do encounter a small edge case: when our `child' character is \texttt{:na}. This, as mentioned in the instructions, is a dummy character and thus, we discard. Commence Elixir pattern matching
\begin{lstlisting}[language=Elixir]
def decode([32 | rest], {:node, :na, _, _}, root, _, message) do
  decode(rest, root, root, "", message)
end    
\end{lstlisting}

\section{The encoder}
The encoder is very trivial to implement. Recursively loop through the plaintext and replace each character with it's morse code equivalent followed by a whitespace. 

\section{Final answers}
The final two texts are
\begin{equation*}
    \texttt{all your bases are belong to us}
\end{equation*}
and
\begin{equation*}
    \texttt{https://www.youtube.com/watch?v=d\%51w4w9\%57g\%58c\%51}
\end{equation*}
which is a link Rick Astley's hit song Never Gonna Give You Up (how dare you rickroll us like that?!).
\begin{figure}[h]
    \centering
    \includegraphics[width=0.7\textwidth]{img/meme}
    \caption{My reaction}
\end{figure}
